\chapter{Conclusiones}

\paragraph{Tiempo de ejecución}

\begin{itemize}
    \item El algoritmo de \emph{Lehmer} tiende a un tiempo de ejecución de $500 \mu s$, para cualquier cantidad de bits.
    \item El tiempo de ejecución del resto de algortimos es directamente proporcional al número de bits.
    \item El algoritmo de \emph{Euclides Binario} presenta mayor inestabilidad(\emph{picos en el grafico}) en el tiempo de ejecución. 
\end{itemize}

\paragraph{Número de iteraciones}

\begin{itemize}                    
    \item EL algoritmo de \emph{Dijkstra(sin modificar)} tiene  el mayor número de iteraciones.
    \item El algoritmo de \emph{Lehmer} sus iteraciones son independientes del número de bits.
    \item El resto de algoritmos son directamente proporcionales al número de bits.
\end{itemize} 


\paragraph{Acumulación de variables}

\begin{itemize}                    
  \item Los algoritmos son independientes en cuanto a la acumulación de varibles.
  \item La mayor cantidad de variables esta reflejado en el algoritmo de Dijkstra.
  \item La menor cantidad de variables esta en el Euclides Binario.
\end{itemize} 
