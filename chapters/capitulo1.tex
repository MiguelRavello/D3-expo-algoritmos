\chapter{Algoritmo de Dijkstra}
\section{Definición}
Dijkstra tiene el siguiente algoritmo:
\begin{equation*}
GCD(m,n)=
 \begin{cases}
    \quad \qquad m & \text{Si $m=n$}\\
    GCD(m-n,n) & \text{Si $m>n$}\\
    GCD(m,n-m) & \text{Si $m<n$}
 \end{cases}
\end{equation*}

Se propone los siguientes puntos \cite{Gcd2011}:
\begin{enumerate}
 \item \textquestiondown Por qu\'e? $d=GCD(m,n)=GCD(m-n,n)$
 \item Si $ d\mid m\: ,\: d\mid n \rightarrow \: d\mid (m-n)$ 
\end{enumerate}
Para el punto 2:
\begin{equation*}
 \begin{align}
  d&=GCD(m,n)\: \rightarrow \: d\mid m \:\&\:d\mid n\\
  if\:&m>n:\\
  m&=dq_1 \: \& \: n=dq_2 \\
  \frac{m}{n} &=\frac{q_1}{q_2} \: \rightarrow \: q_1>q_2
 \end{align}
\end{equation*}
Restando $m-n$:
\begin{equation}
 m-n=d(q_1-q_2)\: \rightarrow \: d\mid (m-n)
\end{equation}
Dado que tenemos:
\begin{equation}
 d \mid (m-n) \:\:\&\:\: d\mid n \rightarrow \:d\mid GCD(m-n,n) \\
\end{equation}
Para el punto 1 necesitamos tener la siguiente estructura:
\begin{itemize}
 \item $a\mid b \:\&\:b\mid a\rightarrow\:a=b$
\end{itemize}
Partiendo de la divisibilidad:
\begin{equation*}
 \begin{align}
  d=&GCD(m,n) \rightarrow d\mid m \:\&\: d\mid n \\
  s=&GCD(m-n,n) \rightarrow s\mid(m-n) \:&\: s\mid n \\
 \end{align}
\end{equation*}
\begin{equation*}
 \begin{align}
  m-n&=sq_3\\
  n&=sq_4\\
  m&=s(q_3+q_4)\\
  m&=sq_5 \rightarrow s\mid m
 \end{align}
\end{equation*}
Dado que:
\begin{equation}
  s \mid m \:\:\&\:\: s\mid n \rightarrow \:s\mid GCD(m,n) \\
\end{equation}
Usando el concepto de la combinaci\'on lineal:
\begin{equation*}
 \begin{align}
  d=GCD(m,n) &\rightarrow d\mid m \:\&\: d\mid n \\
  d&=mx+ny\\
  s=GCD(m-n,n) &\rightarrow s=(m-n)x+ny \\
  s=mx+n(y-x) &\rightarrow s\mid m \:\&\: s\mid n 
 \end{align}
\end{equation*}
\begin{equation}
 s\mid GCD(m,n)
\end{equation}
Para hacer uso de la esta propiedad: 
\begin{equation*}
 a\mid b \:\&\:b\mid a\rightarrow\:a=b
\end{equation*}
Usaremos (2) y (4):
\begin{equation*}
 \begin{align}
  d&=GCD(m,n) \:\&\: s=GCD(m-n,n)\\
  d&\mid GCD(m-n,n)\\
  s&\mid GCD(m,n) \\
  d&\mid s \:\&\:s\mid d\rightarrow\:d=s 
 \end{align}
\end{equation*}
Por lo tanto:
\begin{equation}
 GCD(m-n,n)=GCD(m,n)
\end{equation}
Para el caso de $m<n$ por extenci\'on se repiten los pasos anteriores y obtenemos:
\begin{equation}
 GCD(m,n-m)=GCD(m,n)
\end{equation}

\section{Implementaci\'on}
\subsection{Implementaci\'on basica:}
Es tomada directamente del algoritmo de Dijkstra sin modificaci\'on:
\begin{lstlisting}[language=C++]
#include<NTL/ZZ.h>
#include<omp.h>
#include<cstdio>
using namespace std;
using namespace NTL;
typedef ZZ nat;
nat mcd(nat &m,nat &n){
    if(m==n)
        return m;
    else if(m>n)
        return mcd(m-n,n);
    else if(n>m)
        return mcd(m,n-m);
}
int main(){
    printf("- Simply count elapsed time (CountTime) -\n");
    const double startTime = omp_get_wtime();
    nat a,b,c;
    a=4294967295;
    b=3294967290;
    int k,q;
    SetBit(a,31); //Seteamos a 32
    SetBit(b,31);
    k=NumBits(a); //Verificamos los bits usados
    q=NumBits(b);
    c=mcd(a,b);
    cout<<"mcd: "<<c<<endl;
    cout<<"los bits de a: "<<k<<endl
    cout<<"los bits de b: "<<q<<endl;
    cout<<"los bytes de a: "<<sizeof(a)<<endl;
    cout<<"los bytes de b: "<<sizeof(b)<<endl;
    const double endTime = omp_get_wtime();
    printf("Duration = %lf seconds\n", (endTime-startTime));
    printf("-------------------------------------------\n");
    return 0;
}
\end{lstlisting}
\subsection{Seguimiento del algoritmo:}
En la siguiente tabla vemos las restas sucesivas entre $m\:y\:n$, donde $i$ es el numero de iteraciones hechas.
\begin{longtable}{c|c|c|c}
% aquí añadimos el encabezado de la primera hoja.
\hline
i & m & n & GCD(m,n) \\
\hline \hline
\endfirsthead
% aquí añadimos el encabezado del resto de hojas.
\hline
i & m & n & GCD(m,n) \\
\hline \hline
\endhead
% aquí añadimos el fondo de todas las hojas, excepto de la última.
\multicolumn{4}{c}{Sigue en la página siguiente.}
\endfoot
% aquí añadimos el fondo de la última hoja.
\endlastfoot
% aquí añadimos el cuerpo de la tabla.
1&4294967295&3294967290&GCD(m,n)\\\hline
2&1000000005&3294967290&GCD(m,n)\\\hline
3&1000000005&2294967285&GCD(m,n)\\\hline
4&1000000005&1294967280&GCD(m,n)\\\hline
5&1000000005&294967275&GCD(m,n)\\\hline
6&705032730&294967275&GCD(m,n)\\\hline
7&410065455&294967275&GCD(m,n)\\\hline
8&115098180&294967275&GCD(m,n)\\\hline
9&115098180&179869095&GCD(m,n)\\\hline
10&115098180&64770915&GCD(m,n)\\\hline
11&50327265&64770915&GCD(m,n)\\\hline
12&50327265&14443650&GCD(m,n)\\\hline
13&35883615&14443650&GCD(m,n)\\\hline
14&21439965&14443650&GCD(m,n)\\\hline
15&6996315&14443650&GCD(m,n)\\\hline
16&6996315&7447335&GCD(m,n)\\\hline
17&6996315&451020&GCD(m,n)\\\hline
18&6545295&451020&GCD(m,n)\\\hline
19&6094275&451020&GCD(m,n)\\\hline
20&5643255&451020&GCD(m,n)\\\hline
21&5192235&451020&GCD(m,n)\\\hline
22&4741215&451020&GCD(m,n)\\\hline
23&4290195&451020&GCD(m,n)\\\hline
24&3839175&451020&GCD(m,n)\\\hline
25&3388155&451020&GCD(m,n)\\\hline
26&2937135&451020&GCD(m,n)\\\hline
27&2486115&451020&GCD(m,n)\\\hline
28&2035095&451020&GCD(m,n)\\\hline
29&1584075&451020&GCD(m,n)\\\hline
30&1133055&451020&GCD(m,n)\\\hline
31&682035&451020&GCD(m,n)\\\hline
32&231015&451020&GCD(m,n)\\\hline
33&231015&220005&GCD(m,n)\\\hline
34&11010&220005&GCD(m,n)\\\hline
35&11010&208995&GCD(m,n)\\\hline
36&11010&197985&GCD(m,n)\\\hline
37&11010&186975&GCD(m,n)\\\hline
38&11010&175965&GCD(m,n)\\\hline
39&11010&164955&GCD(m,n)\\\hline
40&11010&153945&GCD(m,n)\\\hline
41&11010&142935&GCD(m,n)\\\hline
42&11010&131925&GCD(m,n)\\\hline
43&11010&120915&GCD(m,n)\\\hline
44&11010&109905&GCD(m,n)\\\hline
45&11010&98895&GCD(m,n)\\\hline
46&11010&87885&GCD(m,n)\\\hline
47&11010&76875&GCD(m,n)\\\hline
48&11010&65865&GCD(m,n)\\\hline
49&11010&54855&GCD(m,n)\\\hline
50&11010&43845&GCD(m,n)\\\hline
51&11010&32835&GCD(m,n)\\\hline
52&11010&21825&GCD(m,n)\\\hline
53&11010&10815&GCD(m,n)\\\hline
54&195&10815&GCD(m,n)\\\hline
55&195&10620&GCD(m,n)\\\hline
56&195&10425&GCD(m,n)\\\hline
57&195&10230&GCD(m,n)\\\hline
58&195&10035&GCD(m,n)\\\hline
59&195&9840&GCD(m,n)\\\hline
60&195&9645&GCD(m,n)\\\hline
61&195&9450&GCD(m,n)\\\hline
62&195&9255&GCD(m,n)\\\hline
63&195&9060&GCD(m,n)\\\hline
64&195&8865&GCD(m,n)\\\hline
65&195&8670&GCD(m,n)\\\hline
66&195&8475&GCD(m,n)\\\hline
67&195&8280&GCD(m,n)\\\hline
68&195&8085&GCD(m,n)\\\hline
69&195&7890&GCD(m,n)\\\hline
70&195&7695&GCD(m,n)\\\hline
71&195&7500&GCD(m,n)\\\hline
72&195&7305&GCD(m,n)\\\hline
73&195&7110&GCD(m,n)\\\hline
74&195&6915&GCD(m,n)\\\hline
75&195&6720&GCD(m,n)\\\hline
76&195&6525&GCD(m,n)\\\hline
77&195&6330&GCD(m,n)\\\hline
78&195&6135&GCD(m,n)\\\hline
79&195&5940&GCD(m,n)\\\hline
80&195&5745&GCD(m,n)\\\hline
81&195&5550&GCD(m,n)\\\hline
82&195&5355&GCD(m,n)\\\hline
83&195&5160&GCD(m,n)\\\hline
84&195&4965&GCD(m,n)\\\hline
85&195&4770&GCD(m,n)\\\hline
86&195&4575&GCD(m,n)\\\hline
87&195&4380&GCD(m,n)\\\hline
88&195&4185&GCD(m,n)\\\hline
89&195&3990&GCD(m,n)\\\hline
90&195&3795&GCD(m,n)\\\hline
91&195&3600&GCD(m,n)\\\hline
92&195&3405&GCD(m,n)\\\hline
93&195&3210&GCD(m,n)\\\hline
94&195&3015&GCD(m,n)\\\hline
95&195&2820&GCD(m,n)\\\hline
96&195&2625&GCD(m,n)\\\hline
97&195&2430&GCD(m,n)\\\hline
98&195&2235&GCD(m,n)\\\hline
99&195&2040&GCD(m,n)\\\hline
100&195&1845&GCD(m,n)\\\hline
101&195&1650&GCD(m,n)\\\hline
102&195&1455&GCD(m,n)\\\hline
103&195&1260&GCD(m,n)\\\hline
104&195&1065&GCD(m,n)\\\hline
105&195&870&GCD(m,n)\\\hline
106&195&675&GCD(m,n)\\\hline
107&195&480&GCD(m,n)\\\hline
108&195&285&GCD(m,n)\\\hline
109&195&90&GCD(m,n)\\\hline
110&105&90&GCD(m,n)\\\hline
111&15&90&GCD(m,n)\\\hline
112&15&75&GCD(m,n)\\\hline
113&15&60&GCD(m,n)\\\hline
114&15&45&GCD(m,n)\\\hline
115&15&30&GCD(m,n)\\\hline
116&15&15&GCD(m,n)\\\hline
\caption{Seguimiento del algoritmo}
\label{ta:morse}
\end{longtable}

\section{Implementaci\'on modificada}
Del seguimiento del algoritmo, como son restas sucesivas, podemos reducir tales restas, con \emph{modulos} sucesivos, siendo esto un Dijkstra-Euclides.
\begin{lstlisting}[language=C++]
#include<NTL/ZZ.h>
#include<omp.h>
#include<cstdio>
using namespace std;
using namespace NTL;
typedef ZZ nat;
nat mcd(nat &m,nat &n){
    if(m==0 || n==0)
        return m+n;
    else if(m>n)
        return mcd(m%n,n);
    else if(n>m)
        return mcd(m,n%m);
}
int main(){
    printf("- Simply count elapsed time (CountTime) -\n");
    const double startTime = omp_get_wtime();
    nat a,b,c;
    a=4294967295;
    b=3294967290;
    int k,q;
    SetBit(a,31); //Seteamos a 32
    SetBit(b,31);
    k=NumBits(a); //Verificamos los bits usados
    q=NumBits(b);
    c=mcd(a,b);
    cout<<"mcd: "<<c<<endl;
    cout<<"los bits de a: "<<k<<endl;
    cout<<"los bits de b: "<<q<<endl;
    cout<<"los bytes de a: "<<sizeof(a)<<endl;
    cout<<"los bytes de b: "<<sizeof(b)<<endl;
    const double endTime = omp_get_wtime();
    printf("Duration = %lf seconds\n", (endTime-startTime));
    printf("-------------------------------------------\n");
    return 0;
}
\end{lstlisting}

\subsection{Seguimiento del codigo}
\begin{table}[!h]
\label{tablax}
\begin{center}
\begin{tabular}{|c|c|c|}
\hline 
i&m&n \\
\hline
1&4294967295&3294967290\\\hline
2&1000000005&3294967290\\\hline
3&1000000005&294967275\\\hline
4&115098180&294967275\\\hline
5&115098180&64770915\\\hline
6&50327265&64770915\\\hline
7&50327265&14443650\\\hline
8&6996315&14443650\\\hline
9&6996315&451020\\\hline
10&231015&451020\\\hline
11&231015&220005\\\hline
12&11010&220005\\\hline
13&11010&10815\\\hline
14&195&10815\\\hline
15&195&90\\\hline
16&15&90\\\hline
17&15&0\\\hline
\end{tabular}
\end{center}
\caption{Iteraciones en Dijkstra-Euclides}
\end{table}
En ambos casos se obtiene el mismo $GCD(m,n)=15$, comparando de solo de estas dos implementaciones el tiempo de ejecuci\'on:
\begin{table}[!h]
\label{tablax}
\begin{center}
\begin{tabular}{|c|c|c|}
\hline 
i&Modificado&Dijkstra \\
\hline
1&$210\mu s$&$208\mu$ \\ \hline
2&$208\mu s$&$215\mu$ \\ \hline
3&$237\mu s$&$240\mu$ \\ \hline
4&$208\mu s$&$154\mu$ \\ \hline
5&$235\mu s$&$216\mu$ \\ \hline
6&$209\mu s$&$219\mu$ \\ \hline
7&$206\mu s$&$256\mu$ \\ \hline
8&$235\mu s$&$251\mu$ \\ \hline
9&$203\mu s$&$237\mu$ \\ \hline
10&$206\mu s$&$267\mu$ \\ \hline
\end{tabular}
\end{center}
\caption{Comparaci\'on de tiempos de ejecuci\'on}
\end{table}
Relativamente el modificado es mejor que el Dijkstra normal, por lo que los datos del Dijkstra-Euclides se usaran para comparar con los otros algoritmos.