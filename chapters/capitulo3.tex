\chapter{Algoritmo Binario de Euclides}
\section{Definición}
El algoritmo binario de euclides opera con la misma idea , de cambiar mcd(a,b) por el mcd de dos numeros eventualmente pequeños , solo que esta vez solo restaremos y dividimos.\\
Este algoritmo opera con los siguientes teoremas:
\begin{itemize}
 \item Si a,b son pares, $mcd(a,b) = 2 mcd(a/2,b/2)$.
 \item Si a es par y b impar o viceversa, $mcd(a,b) = mcd(a/2,b)\: o\: mcd(a, b/2)$.
 \item Si a,b son impares , $mcd(a,b) = mcd(|a-b|/2, r)$, donde $r = min{a,b}$;
\end{itemize}
\begin{lstlisting}[language=C++]
Algoritmo binario para el mcd
Datos: a, b e Z, a >= 0, b > 0
Salida: mcd ( a, b )
    g = 1;
    while a mod 2 = 0 And b mod 2 = 0 do
         a = quo ( a, 2 ) , b = quo ( b, 2 ) ;
         g = 2<<g //removiendo potencias de 2
    while a  = 0 do // Ahora, a o b es impar
      if a mod 2 = 0 then
          a = quo ( a, 2 )
      else if b mod 2 = 0 then
          b = quo ( b, 2 )
      else ; // ambos impares  
          t = quo (| a − b | , 2 ) ;
          if a >= b then ; // reemplazamos máx { a, b } con quo (| a - b | , 2 )
             a = t
          else
             b = t
    return g*b;  
\end{lstlisting}
\section{Implementación}
\begin{lstlisting}[language=C++]
#include<NTL/ZZ.h>
#include<omp.h>
#include<cmath>
using namespace std;
using namespace NTL;

using nat = ZZ;

nat dividir2(nat);
nat multiplicar2(nat);
nat GCD(nat &, nat &);
int main(){
   const double startTime = omp_get_wtime();
   nat a, b,c;
   a = 768454923;
   b = 542167814;
   SetBit(a,63);
   SetBit(b,63);
   c = GCD(a,b);
   cout << c << endl;
   const double endTime = omp_get_wtime();
   printf("Duration = %lf seconds\n", (endTime - startTime));
   return 0;
}
nat dividir2(nat  a){
   return a >> 1 ;
}
nat multiplicar2(nat a){
   return a << 1;
}
nat GCD(nat &a, nat &b){
   nat result, t,g;
   result = 0,g = 1,t = 0;
   while(( a % 2 == 0) && (b % 2 == 0)){
      a = dividir2(a);
      b = dividir2(b);
      g = multiplicar2(g);
   }
   while( a != 0 ){
      if (a % 2 == 0)
         a = dividir2(a);
      else{
         if(b % 2 == 0)
            b = dividir2(b);
         else{
            t = dividir2( abs(a-b) );
            if (a >= b)
               a = t;
            else
               b = t;
         }
      }
   }
   result = g * b;
   return result;
}
\end{lstlisting}
\section{Seguimiento del algoritmo}
\begin{longtable}{c|c|c|c}
% aquí añadimos el encabezado de la primera hoja.
\hline
i & a &  b & t \\
\hline \hline
\endfirsthead
% aquí añadimos el encabezado del resto de hojas.
\hline
i & a & b & t \\
\hline \hline
\endhead
% aquí añadimos el fondo de todas las hojas, excepto de la última.
\multicolumn{4}{c}{Sigue en la página siguiente.}
\endfoot
% aquí añadimos el fondo de la última hoja.
\endlastfoot
1&4294967295&3294967290&0\\\hline
2&4294967295&1647483645&0\\\hline
3&1323741825&1647483645&1323741825\\\hline
4&1323741825&161870910&161870910\\\hline
5&1323741825&80935455&161870910\\\hline
6&621403185&80935455&621403185\\\hline
7&270233865&80935455&270233865\\\hline
8&94649205&80935455&94649205\\\hline
9&6856875&80935455&6856875\\\hline
10&6856875&37039290&37039290\\\hline
11&6856875&18519645&37039290\\\hline
12&6856875&5831385&5831385\\\hline
13&512745&5831385&512745\\\hline
14&512745&2659320&2659320\\\hline
15&512745&1329660&2659320\\\hline
16&512745&664830&2659320\\\hline
17&512745&332415&2659320\\\hline
18&90165&332415&90165\\\hline
19&90165&121125&121125\\\hline
20&90165&15480&15480\\\hline
21&90165&7740&15480\\\hline
22&90165&3870&15480\\\hline
23&90165&1935&15480\\\hline
24&44115&1935&44115\\\hline
25&21090&1935&21090\\\hline
26&10545&1935&21090\\\hline
27&4305&1935&4305\\\hline
28&1185&1935&1185\\\hline
29&1185&375&375\\\hline
30&405&375&405\\\hline
31&15&375&15\\\hline
32&15&180&180\\\hline
33&15&90&180\\\hline
34&15&45&180\\\hline
35&15&15&15\\\hline
\caption{Seguimiento del algoritmo}
\label{ta:m}
\end{longtable}